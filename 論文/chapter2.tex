\chapter{背景知識}

\section{國際化}
國際化\cite{internationalization},英文讀作Internationalization,簡稱i18n,18代表i和n之間的18個英文字母。國際化是開發軟體時,將軟體本身和特定語言、地區脫鉤的一個過程,除了可以滿足不同的地區、文化的大眾需求,移植軟體到不同的語言環境時,也不須改變內部程式的實作。

\section{自動化驗收測試}
驗收測試\cite{se},英文讀作Acceptance Testing,是一種站在使用者的立場,去檢驗一個真實存在的系統,是否滿足使用者需求與預期的測試方法。

而透過撰寫自動化測試腳本\cite{AT},如今我們可以執行更符合成本,且更加精確的自動化驗收測試。改善了過往手動執行驗收測試時,存在的人為操作誤差、系統問題無法即時呈現、成本較高等問題。
\section{Robot Framework}
Robot Framework\cite{rf}\cite{rfguide}是一個開源的框架語言,可以用來執行自動化驗收測試或機器人自動化,核心框架是由Python\cite{python}編寫而成,測試者可以使用Python或Java擴充其函式庫。其特色是擁有簡單的語法,以及容易理解的原生關鍵字(Keyword),測試者可以視需求使用並包裝成更接近自然語言的關鍵字。

\subsection{Robot Framework測試腳本}
一個Robot Framework 的測試腳本基本上由三個區塊構成,分別是:
\begin{itemize}
    \item[1.]Settings: 包含了使用到的library與resource file,也可以將Test Setup(測試腳本執行前要做的動作)、Test Teardown(測試腳本執行後要做的動作)定義於此。
    \item[2.]Test Cases: 在此測試者可以為各項想要驗證的使用者需求,撰寫核心的測試腳本
    \item[3.]Keywords: 如果測試腳本即將使用並非Robot Framework的原生關鍵字,或未被定義於其他resource file中,測試者可以在此處撰寫出新的關鍵字去達到測試目的。
\end{itemize}

\subsection{Robot Framework測試報表}
在測試腳本運行結束後,Robot Framework會產生出一份測試報表,記錄了整體的執行狀況,包含測試通過或失敗、腳本執行時間,並透過可展開的階層式圖表,方便測試者去追蹤到當前出現問題的關鍵字,進而修復錯誤。

\section{XPath}
XPath\cite{xpath},全名為 XML Path Language,可以用來定位XML檔案中某節點所處在的位置。在使用Robot Framework撰寫的網頁自動化驗收測試中,我們便時常需要藉助XPath,去找到並確定畫面上某元件的位置,以對其狀態進行測試。