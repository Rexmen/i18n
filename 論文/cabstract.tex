\chapter*{摘~~要}
\addcontentsline{toc}{chapter}{中文摘要}

%基本資訊

\noindent
論文名稱:支援多國語言的Robot Framework網頁自動化驗收測試工具的功能改善與擴充\\
頁數:68頁\\
校所別:國立台北科技大學~資訊工程系碩士班\\
畢業時間:一百一十學年度第二學期\\
學位:碩士\\
研究生:林稟宸\\
指導教授:謝金雲、鄭有進教授\\
\hspace*{\fill}\\
\noindent
相關名詞:Robot Framework、XPath、關鍵字(Keyword)、代理關鍵字(Keyword Proxy)、i18n\\
\hspace*{\fill}\\
%
\indent
在現今的國際化社會,一個網頁可能同時擁有多種語言的版本。根據朱峻平的論文「支援多國語言的 Robot Framework 網頁自動化驗收測試」所提出的第一版i18n工具,目前僅能支援七種Robot Framework原生關鍵字的翻譯代理,且也只能對三種XPath內的HTML屬性進行翻譯;本論文將針對以上兩點,對第一版i18n工具進行代理關鍵字的擴充和XPath翻譯邏輯的改善。除此之外,並會提供一詞多譯的圖形化使用者介面讓使用者去選擇期望的翻譯,以解決一詞多譯的問題。最後,將i18n工具包裝成為可以透過pip安裝的Python模組。
