\chapter*{摘~~要}
\addcontentsline{toc}{chapter}{中文摘要}

%基本資訊

\noindent
論文名稱:支援多國語言的Robot Framework網頁自動化驗收測試工具的功能改善與擴充\\
頁數:55頁\\
校所別:國立台北科技大學~資訊工程系碩士班\\
畢業時間:一百一零學年度第二學期\\
學位:碩士\\
研究生:林稟宸\\
指導教授:謝金雲、鄭有進教授\\
\hspace*{\fill}\\
\noindent
相關名詞:Robot Framework、xpath、關鍵字(Keyword)、代理關鍵字(Keyword Proxy)、i18n\\
\hspace*{\fill}\\
%
\indent
以往基於Robot Framework的網頁驗收測試,同一份測試腳本的測試對象往往只限於一種語言的網頁,
但現今的國際化社會,隨著商業模式的改變,一個網頁可能有多種語言的版本給不同國家的人使用。
為了做出類似或相同的測試目的,測試人員需要撰寫更多重複性高的測試腳本,如此一來,測試的成本便會大大增加。

根據朱峻平的論文: ”支援多國語言的Robot Framework 網頁自動化驗收測試” (i18n),
透過使用者提供的JSON格式多國語言網頁翻譯檔,建立出一份單字對應翻譯的翻譯路徑檔,
並在程式中定義翻譯的邏輯以及代理關鍵字,即可在不改動Robot Framework 測試腳本的情況下,
完成多國語言網頁自動化驗收測試。

然而,現在的i18n工具仍然存在著許多待改善之處,會造成使用上的困難。
分別是: 支援的Robot Framework原生關鍵字只有七種、翻譯邏輯只支援三種HTML屬性、
一詞多譯無法讓使用者選擇期望的翻譯、該工具尚未發佈並支援安裝。因此,本論文將針對上述幾點待改善的地方,
進行功能上的補強,希望藉此讓i18n工具在未來變得更易於使用。

