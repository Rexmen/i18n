\chapter*{ABSTRACT}
\addcontentsline{toc}{chapter}{ABSTRACT}

%基本資訊

\noindent
Title: Further Improvement and Extension of an Automated Web Testing Tool 
with Robot Framework to Support Internationalization\\
Pages: 55\\
School: National Taipei University of Technology\\
Department: Computer Science and Information Engineering\\
Time: June,2021\\
Degree: Master\\
Researcher: Bing-Chen Lin\\
Advisor: C.-Y. Hsieh and Y.C.Cheng\\
\hspace*{\fill}\\
Related terms: Robot Framework, xpath, Keyword, Keyword Proxy, I18n\\
\hspace*{\fill}\\
%
\indent
In the past, the test target of an acceptance test script with Robot Framework 
was often a  web page with single language. However, 
nowadays a web page may have multiple language versions for people 
who live in different countries. Therefore, test engineers often need to 
write test scripts which are highly repetitive, 
and the cost of testing would also increase.

According to the thesis “Internationalization Support for Automated Web Testing
with Robot Framework”(i18n) written by Chun-Ping Chu, i18n tool can build 
the mapping between words and their translations by JSON format translation 
files given by user. Besides, through the translation logic and 
keyword proxies implemented in i18n tool, we can run web tests which have 
same movements but different languages by a same test script without changing a word.

However, the i18n tool still have many flaws to be improved currently; namely, 
supporting only 7 Robot Framework native keywords, limiting to 3 HTML attributes as translate targets, 
not supporting user choices to translations of words, not supporting installation by pip.

Therefore, I will improve the flaws of the i18n tool mentioned above. 
Hoping that the i18n tool would be better and more user-friendly in the future.

