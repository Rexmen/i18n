\chapter{緒論}

\section{研究背景與動機}
根據朱峻平的論文:“支援多國語言的Robot Framework 網頁自動化驗收測試”(i18n)\cite{i18n},透過使用者提供的JSON格式\cite{json}多國語言網頁翻譯檔,建立出一份單字對應翻譯的翻譯路徑檔,並在程式中定義翻譯的邏輯以及代理關鍵字,即可在不改動Robot Framework測試腳本的情況下,完成多國語言網頁自動化驗收測試。

然而,現在的i18n\cite{internationalization}工具仍然存在著許多待改善之處,例如: 

\begin{itemize}
\item[1.] 目前i18n工具只支援7種Robot Framework\cite{rf}原生關鍵字,如果未來測試腳本使\\
用到其他未支援的Robot Framework原生關鍵字,便會發生錯誤。
\item[2.] 程式目前的翻譯對象僅限於網頁上的text(), normalize-space(),以及@title三種\\
HTML屬性。屆時,若使用者的測試腳本是用沒有列舉出來的屬性撰寫,例\\
如:@placeholder,@arial-label等等,便會出錯。
\item[3.] 測試腳本執行期間,若遭遇一詞多譯的情況,目前i18n工具僅於報表上列出該待\\
翻譯詞的可能翻譯有哪些,並從中選取會使測試通過的翻譯詞當作當前翻譯。使\\
用者無法自己選擇期望的翻譯去跑測試腳本。此外,假如存在多個翻譯詞都會通\\
過此測試(可能是因為XPath\cite{xpath}\cite{stablexpath}使用contains語法,使得畫面上要驗證的資訊只\\
要包含於翻譯詞,測試就會通過;又或者畫面上剛好有多個元件符合翻譯後的測\\
試腳本),翻譯過後腳本的測試對象,就會偏離了使用者原先的預期。
\item[4.] 目前i18n工具尚未包裝成可以讓使用者直接安裝後使用的擴充工具,停留在使用\\
者必須將github上的專案clone下來,並且在該專案上開發測試腳本的階段。如\\
此會造成使用者的許多不便。
\end{itemize}

有鑑於此,本論文將對現有的i18n工具進行功能的改善與擴充。

\section{研究目標}
本論文旨在改善i18n工具上述提及的四項缺陷,希望透過以下努力可以讓此I18n\\
工具在未來執行多國語言網頁自動化驗收測試\cite{se}\cite{testduo}時,更易於使用。

實作的目標分別列舉如下:
\begin{itemize}
\item[1.] 擴充完剩下的Robot Framework原生關鍵字代理,使得所有常用原生關鍵字的參\\
數部分都能正確的被翻譯。
\item[2.] 透過“負面表列法”,將XPath中確定不會執行翻譯的屬性(例如:@id , @class等等)儲存於list中。若腳本執行到當下的XPath需要被翻譯,且其中有屬性不在list中,則利用這些屬性們生成一個新的翻譯規則,再執行翻譯。
\item[3.] 執行完含有一詞多譯的測試腳本後,如果測試通過,便開啟一個圖形化介面,記錄了執行翻譯時當下關鍵字的參數組合,並顯示所有可能的翻譯詞,讓使用者可以從中去選擇,並產生一個設定檔。之後再次執行測試腳本時,i18n工具便會根據設定檔的內容去選擇適當的翻譯詞,同時消除報表上的warning提示。 
\item[4.] 將i18n工具包裝成為可以pip\cite{PIP} install 的library。
\end{itemize}

\section{論文組織架構}
本論文分為五個章節,第一章介紹研究背景與動機,以及期望達成的研究目標。第二章介紹本論文相關的背景知識。第三章探討本論文的研究方法與實作,詳述擴充代理關鍵字的流程、改善翻譯邏輯的完整思路,以及如何藉由圖形化使用者介面去改善i18n工具目前遇到一詞多譯時的處理方法。第四章展示實際的測試案例,呈現做了上述改動後,使用者執行多國語言網頁自動化驗收測試時,會得到什麼結果,並可自行與第一版i18n工具\cite{i18n}做比較。第五章則是結論以及未來展望,概述本論文的成果以及尚且存在的使用限制和待改善之處。